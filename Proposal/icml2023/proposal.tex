%%%%%%%% ICML 2023 EXAMPLE LATEX SUBMISSION FILE %%%%%%%%%%%%%%%%%

\documentclass{article}

% Recommended, but optional, packages for figures and better typesetting:
\usepackage{microtype}
\usepackage{graphicx}
\usepackage{subfigure}
\usepackage{booktabs} % for professional tables

% hyperref makes hyperlinks in the resulting PDF.
% If your build breaks (sometimes temporarily if a hyperlink spans a page)
% please comment out the following usepackage line and replace
% \usepackage{icml2023} with \usepackage[nohyperref]{icml2023} above.
\usepackage{hyperref}


% Attempt to make hyperref and algorithmic work together better:
\newcommand{\theHalgorithm}{\arabic{algorithm}}

% Use the following line for the initial blind version submitted for review:
\usepackage{icml2023}

% If accepted, instead use the following line for the camera-ready submission:
% \usepackage[accepted]{icml2023}

% For theorems and such
\usepackage{amsmath}
\usepackage{amssymb}
\usepackage{mathtools}
\usepackage{amsthm}

% if you use cleveref..
\usepackage[capitalize,noabbrev]{cleveref}

%%%%%%%%%%%%%%%%%%%%%%%%%%%%%%%%
% THEOREMS
%%%%%%%%%%%%%%%%%%%%%%%%%%%%%%%%
\theoremstyle{plain}
\newtheorem{theorem}{Theorem}[section]
\newtheorem{proposition}[theorem]{Proposition}
\newtheorem{lemma}[theorem]{Lemma}
\newtheorem{corollary}[theorem]{Corollary}
\theoremstyle{definition}
\newtheorem{definition}[theorem]{Definition}
\newtheorem{assumption}[theorem]{Assumption}
\theoremstyle{remark}
\newtheorem{remark}[theorem]{Remark}

% Todonotes is useful during development; simply uncomment the next line
%    and comment out the line below the next line to turn off comments
%\usepackage[disable,textsize=tiny]{todonotes}
\usepackage[textsize=tiny]{todonotes}


% The \icmltitle you define below is probably too long as a header.
% Therefore, a short form for the running title is supplied here:
\icmltitlerunning{Submission and Formatting Instructions for ICML 2023}

\begin{document}

\twocolumn[
\icmltitle{Optimisitc initialization of parameterized value functions\\ using a neural network}

% It is OKAY to include author information, even for blind
% submissions: the style file will automatically remove it for you
% unless you've provided the [accepted] option to the icml2023
% package.

% List of affiliations: The first argument should be a (short)
% identifier you wi ll use later to specify author affiliations
% Academic affiliations should list Department, University, City, Region, Country
% Industry affiliations should list Company, City, Region, Country

% You can specify symbols, otherwise they are numbered in order.
% Ideally, you should not use this facility. Affiliations will be numbered
% in order of appearance and this is the preferred way.
\icmlsetsymbol{equal}{*}

\begin{icmlauthorlist}
\icmlauthor{Alireza Azimi}{equal, dept}
\icmlauthor{Haruto Tanaka}{equal, dept}
\icmlauthor{Henry Du}{equal, dept}
\icmlauthor{Mashfiq Shahriar Zaman}{equal, dept}
%\icmlauthor{}{sch}
%\icmlauthor{}{sch}
%\icmlauthor{}{sch}
\end{icmlauthorlist}

\icmlaffiliation{dept}{Department of Computing Science, University of Alberta}


% \icmlcorrespondingauthor{Firstname1 Lastname1}{first1.last1@xxx.edu}
% \icmlcorrespondingauthor{Firstname2 Lastname2}{first2.last2@www.uk}

% You may provide any keywords that you
% find helpful for describing your paper; these are used to populate
% the "keywords" metadata in the PDF but will not be shown in the document
\icmlkeywords{Machine Learning, Reinforcement Learning, Deep Reinforcement Learning}

\vskip 0.3in
]

% this must go after the closing bracket ] following \twocolumn[ ...

% This command actually creates the footnote in the first column
% listing the affiliations and the copyright notice.
% The command takes one argument, which is text to display at the start of the footnote.
% The \icmlEqualContribution command is standard text for equal contribution.
% Remove it (just {}) if you do not need this facility.

%\printAffiliationsAndNotice{}  % leave blank if no need to mention equal contribution
\printAffiliationsAndNotice{\icmlEqualContribution} % otherwise use the standard text.

\begin{abstract}
Optimistic initialization of value functions \citep{Ando2007} is a popular approach to exploration in tabular reinforcement learning. However, deep reinforcement learning seldom undergoes thorough analysis. In this study, we investigate the aforementioned issue by employing a parametrized value function that utilizes linear neural networks. Furthermore, we conduct a comparative analysis between our findings and those obtained from a widely adopted learning methodology.
\end{abstract}

\section{Introduction}
The persistent challenge of balancing exploration and exploitation is a fundamental challenge encountered by each reinforcement learning algorithm. Since this challenge arises everywhere and impacts the overall performance of algorithms, various methods to balance exploration or exploitation have been proposed. Among those approaches, one of the most fundamental and flexible is an optimistic initialization of the value functions. By simply setting the initial values greater than the reward maxima, one can force the agents to explore every state(-action) pair at least once at the early stage. The impact of optimistic initialization is noteworthy, particularly in scenarios involving tabular data, despite its straightforward nature. Although this simple technique is proven to provide tremendous advantages in a tabular context, its effect on deep nonlinear function approximation is yet to be discovered. This project aims to provide 1. a basic method to optimistically initialize the deep network; 2. a simple fix on the implementation so that optimistic values remain effective after a few gradient steps; and 3. empirical analysis.


% \subsection{Submitting Papers}

%Authors may submit to ICML substantially different versions of journal papers
%that are currently under review by the journal, but not yet accepted
%at the time of submission.
\medskip
\section{Background}


\section{Research Question}
This study aims to address the following three-part question:
Can a parameterized value function using a neural network be optimistically initialized using reward signal shift and normalization? How will the removal of the bias term from the neural network affect optimistic initialization? In a value estimation experiment, how will this initialization compare to it's tabular counterpart using SARSA and semi-gradient SARSA with the same reward signal shift?



\section{Experimental Design}
In this research, we will utilize the minigrid library \citep{Ando2005} to conduct our experiments in three stationary environments Crossingenv, Distshiftenv, and Lavagapenv.\\
The value estimation algorithm in our tabular setting is SARSA and for the parameterized setting we will be using
semi-gradient SARSA \citep{Ando2008}, using an $\epsilon$-greedy behavior \citep{Ando2006} policy.\\
The API of the minigrid library offers a discrete action space, which is supported by the encoding of states in an image format for the parameterized setting and in a coordinate format for the tabular configuration.\\
We will perform SARSA using an $\epsilon$-greedy policy to estimate the state-action value functions using the agent coordinates while initializing the state-action values optimistically. The concept will be expanded to the parameterized scenario by modifying the neural network's weights in such a way that it produces an optimistic result for all state-action values. The parameterized setting experiment will be replicated by eliminating the bias term from the neural network. Subsequently, the experiments will be readjusted and conducted accordingly. By doing a comparative analysis of the outcomes, we can acquire a more comprehensive understanding of the procedure involved in initializing parameterized value functions, its impact on the predicted return, and the significance of the bias term in relation to performance.\\
In addition, the aforementioned environments were chosen based on their simpler dynamics and smaller state-space, which contributes to reducing the computing cost of our experiment.


% Acknowledgements should only appear in the accepted version.
\section{Contributions}
\textbf{Alireza Azimi:}\\
\textbf{Haruto Tanaka:}\\
\textbf{Henry Du:}\\
\textbf{Mashfiq Shahriar Zaman:}\\

% \section{References}
% To be added later.
% In the unusual situation where you want a paper to appear in the
% references without citing it in the main text, use \nocite
\nocite{Ando2005, Ando2006, Ando2007, Ando2008}

\bibliography{custom}
\bibliographystyle{icml2023}

\end{document}
