%%%%%%%% ICML 2023 EXAMPLE LATEX SUBMISSION FILE %%%%%%%%%%%%%%%%%

\documentclass{article}

% Recommended, but optional, packages for figures and better typesetting:
\usepackage{microtype}
\usepackage{graphicx}
\usepackage{subfigure}
\usepackage{booktabs} % for professional tables

% hyperref makes hyperlinks in the resulting PDF.
% If your build breaks (sometimes temporarily if a hyperlink spans a page)
% please comment out the following usepackage line and replace
% \usepackage{icml2023} with \usepackage[nohyperref]{icml2023} above.
\usepackage{hyperref}


% Attempt to make hyperref and algorithmic work together better:
\newcommand{\theHalgorithm}{\arabic{algorithm}}

% Use the following line for the initial blind version submitted for review:
\usepackage{icml2023}

% If accepted, instead use the following line for the camera-ready submission:
% \usepackage[accepted]{icml2023}

% For theorems and such
\usepackage{amsmath}
\usepackage{amssymb}
\usepackage{mathtools}
\usepackage{amsthm}

% if you use cleveref..
\usepackage[capitalize,noabbrev]{cleveref}

%%%%%%%%%%%%%%%%%%%%%%%%%%%%%%%%
% THEOREMS
%%%%%%%%%%%%%%%%%%%%%%%%%%%%%%%%
\theoremstyle{plain}
\newtheorem{theorem}{Theorem}[section]
\newtheorem{proposition}[theorem]{Proposition}
\newtheorem{lemma}[theorem]{Lemma}
\newtheorem{corollary}[theorem]{Corollary}
\theoremstyle{definition}
\newtheorem{definition}[theorem]{Definition}
\newtheorem{assumption}[theorem]{Assumption}
\theoremstyle{remark}
\newtheorem{remark}[theorem]{Remark}

% Todonotes is useful during development; simply uncomment the next line
%    and comment out the line below the next line to turn off comments
%\usepackage[disable,textsize=tiny]{todonotes}
\usepackage[textsize=tiny]{todonotes}


% The \icmltitle you define below is probably too long as a header.
% Therefore, a short form for the running title is supplied here:
\icmltitlerunning{Submission and Formatting Instructions for ICML 2023}

\begin{document}

\twocolumn[
\icmltitle{Optimisitc initialization of parameterized value functions\\ using a neural network}

% It is OKAY to include author information, even for blind
% submissions: the style file will automatically remove it for you
% unless you've provided the [accepted] option to the icml2023
% package.

% List of affiliations: The first argument should be a (short)
% identifier you wi ll use later to specify author affiliations
% Academic affiliations should list Department, University, City, Region, Country
% Industry affiliations should list Company, City, Region, Country

% You can specify symbols, otherwise they are numbered in order.
% Ideally, you should not use this facility. Affiliations will be numbered
% in order of appearance and this is the preferred way.
\icmlsetsymbol{equal}{*}

\begin{icmlauthorlist}
\icmlauthor{Alireza Azimi}{equal, dept}
\icmlauthor{Haruto Tanaka}{equal, dept}
\icmlauthor{Henry Du}{equal, dept}
\icmlauthor{Mashfique Zaman}{equal, dept}
%\icmlauthor{}{sch}
%\icmlauthor{}{sch}
%\icmlauthor{}{sch}
\end{icmlauthorlist}

\icmlaffiliation{dept}{Department of Computing Science, University of Alberta}


% \icmlcorrespondingauthor{Firstname1 Lastname1}{first1.last1@xxx.edu}
% \icmlcorrespondingauthor{Firstname2 Lastname2}{first2.last2@www.uk}

% You may provide any keywords that you
% find helpful for describing your paper; these are used to populate
% the "keywords" metadata in the PDF but will not be shown in the document
\icmlkeywords{Machine Learning, Reinforcement Learning, Deep Reinforcement Learning}

\vskip 0.3in
]

% this must go after the closing bracket ] following \twocolumn[ ...

% This command actually creates the footnote in the first column
% listing the affiliations and the copyright notice.
% The command takes one argument, which is text to display at the start of the footnote.
% The \icmlEqualContribution command is standard text for equal contribution.
% Remove it (just {}) if you do not need this facility.

%\printAffiliationsAndNotice{}  % leave blank if no need to mention equal contribution
\printAffiliationsAndNotice{\icmlEqualContribution} % otherwise use the standard text.

\begin{abstract}
Optimistic initialization of value functions is a popular approach to exploration
in tabular reinforcement learning. However, it is rarely analyzed
in deep reinforcement learning. We explore this problem through a parameterized
value function using linear neural networks and compare our results to an existing popular 
learning algorithm.
\end{abstract}

\section{Introduction}
Tradeoff between exploration and exploitation, is an eternal problem that any RL algorithms face.
Since this challenge arises everywhere and impacts the overall performance of algorithm, various methods to balance the exploration/exploitation has been proposed.
Among those approaches, one of the most fundamental and flexible approach is an optimistic initialization of the value functions.
By simply setting the initial values greater than the reward maxima, one can enforce the agents to explore every state(-action) pair at least once at the early stage.
Despite of its simplicity, the effect of optimistic initialization is significant, especially on the tabular cases.
Although this simple technique is proven to provide the tremendous advantages in tabular context, its effect on deep nonlinear function approximation is yet to be discovered.
This project aims to provide 1. a basic method to optimistically initialize the deep network, 2. simple fix on the implementation so that optimistic values remain effective after few gradient steps and 3. empirical analysis.

% \subsection{Submitting Papers}

%Authors may submit to ICML substantially different versions of journal papers
%that are currently under review by the journal, but not yet accepted
%at the time of submission.
\medskip
\section{Background}
As an exploration method in RL, \textbf{optimistic initialization} is often used in diverse set of tasks due to its high scalability.
Depending on the task, however, the methods for giving this optimism changes slightly.
One obvious method is to naively set the initial values to an arbitrary constant for all the states.
This method is known to work well in the tabular RL tasks, but it is not guaranteed that it works in other situtations such as function approximation.
Particularly for function approximation, it is almost unable to set the optimism naively without knowing the upper bound of return signals.
To avoid this, Mochado et al (2015) proposed a method to normalize and shift down the reward signals.
This technique allows us to determine an upper bound on any tasks, and helps us to naively initialize the function parameters.
However, the method only assumed the linear function approximator, and not yet tested on DRL algorithms.


\section{Research Question}
In this paper we strive to answer the following 3 part question,
Can a parameterized value function using a neural network be optimistically intialized using reward signal shift and normalization?
How will the removal of the bias term from the neural network affect optimistic initialization?
In a value estimation experiment how will this initialization compare to it's tabular counterpart using 
SARSA and semi-gradient SARSA with the same reward signal shift?
\section{Experimental Design}
We will be considering the minigrid library to run our experiments in three stationary environments. For the purposes of this paper, we will be focusing on the stationary environments
Crossingenv, Distshiftenv and Lavagapenv.\\
The value estimation algorithm in our tabular setting is SARSA and for the parameterized setting we will be using
semi-gradient SARSA, using an $\epsilon$-greedy behavior policy.\\
The mingrid library API provides a discrete action space, along with encoding states in an image format for the 
parameterized setting and a coordinate format for the tabular setting.\\
We will perform SARSA using an epsilon-greedy policy to estimate the state-action value functions using the agent coordinates while initializing the state-action 
values optimistically. We will extend this idea to the parameterized case by adjusting the weights of the neural
network so that it outputs an optimistic value for all state-action values. We will repeat the parameterized
setting experiment by removing the bias term from the neural network and re-adjusting and performing the experiments accordingly.
Comparing the results should give us better insight into the process of initializing paramaterized value functions, it's effect on 
expected return and the siginificance of the bias term in performance.\\
Furthermore, the mentioned environments were selected to due having simpler environment dynamics and a smaller state-space, making our experiment computationally less expensive.


% Acknowledgements should only appear in the accepted version.
\section{Contributions}
\textbf{Alireza Azimi:}\\
\textbf{Haruto Tanaka:}\\
\textbf{Henry Du:}\\
\textbf{Mashfique Zaman:}\\

\section{References}
To be added later.
% In the unusual situation where you want a paper to appear in the
% references without citing it in the main text, use \nocite
\nocite{langley00}

\bibliography{example_paper}
\bibliographystyle{icml2023}

\end{document}
